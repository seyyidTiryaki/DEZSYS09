\documentclass[12pt]{article}

\usepackage[english]{babel}
\usepackage[utf8x]{inputenc}
\usepackage{amsmath}
\usepackage{enumitem}
\usepackage{graphicx}
\usepackage{ulem}
\usepackage{caption}
\usepackage{placeins}
\usepackage[usenames,dvipsnames]{color}
\usepackage[colorinlistoftodos]{todonotes}
\usepackage{listings}
\usepackage{fixltx2e}
\usepackage{scrpage2}
\usepackage{lastpage}
\clearscrheadfoot
\pagestyle{scrheadings}
\usepackage{glossaries}
\usepackage[
top    = 2.75cm,
bottom = 2.00cm,
left   = 2.50cm,
right  = 2.00cm]{geometry}
\setcounter{secnumdepth}{4}


\makeglossaries

\newglossaryentry{glossaryVerweis} {name=abkuerzung, description={Langer Name}}


\begin{document}
\begin{titlepage}
\begin{center}
% Oberer Teil der Titelseite:
\includegraphics[width=0.5\textwidth]{images/logo}\\[1cm]    

\textsc{\LARGE Technologisches Gewerbe Museum}\\[1.5cm]

% Title
\rule{12cm}{1mm}
{ \huge \bfseries  \\\large DezSys\\ \huge Extract, Transform and Load (ETL) with EAI \\[0.4cm] }

\rule{12cm}{1mm}

% Author and supervisor
\noindent 
\vspace{5cm}

\begin{center}
\large
Author: 
Pöcher \textsc{Rene} \&
Tiryaki \textsc{Seyyid}
\end{center}

\vfill

% Bottom of the page
{\large \today}

\end{center}
\end{titlepage}

\tableofcontents


%HEADER AND FOOTER
\pagenumbering{arabic}
\ohead{\headmark}
\automark{section}
\ifoot{@ Authors}
\ofoot{\pagemark ~of \pageref{LastPage}}

\newpage



\section{Working time}
\subsection{Estimated Working time}
\begin{center}
\textbf{Estimated working time}
\end{center}

\begin{table}[h]

\begin{tabular}{|p{0.4\textwidth}|p{0.2\textwidth}|p{0.2\textwidth}|}
\hline
\textbf{Task}    & \textbf{Person}                                               & \textbf{Time in hours                              } \\ \hline \hline
Task1 & \begin{tabular}[c]{c}Coworker1\\ Coworker2\end{tabular} & \begin{tabular}[c]{c}1\\ 1\end{tabular}    \\ \hline 
Task2 & \begin{tabular}[c]{c}Coworker1\\ Coworker2\end{tabular} & \begin{tabular}[c]{c}1\\ 1\end{tabular}    \\ \hline \hline
Total & \begin{tabular}[c]{c}Coworker1\\ Coworker2\end{tabular} & \begin{tabular}[c]{c}2\\2\end{tabular}   \\ \hline 
\textbf{Total Team} & & \textbf{4 hours}  \\ \hline 
\end{tabular}
\caption{Estimated working time}
\end{table}


\section{Identifikation und Beschreibung der EAI Patterns}

Bei der Enterprise Application Integration (EAI) handelt es sich um die Integration großer
Anwendungssysteme (Enterprise Applications). Diese wird in den meisten Fällen nachrichtenbasiert durchgeführt. Ein Integrationsnetz zwischen mehreren Anwendungssystemen lässt
sich durch Kombination verschiedener EAI Patterns erstellen. Dabei wechseln sich die nachrichtenverarbeitenden Teile (Filter) und die Kanäle zwischen den Filtern (Pipes) ab und
bilden somit einen Graphen \cite{EAIUniStuttgart}


\section{Beschreibung der Funktionsweise von Apache Camel}

Apache Camel ist eine freie, regelbasierte Routing- und Konvertierungsengine. Mit Apache Camel kann man Routing- und Konvertierungsregeln deklarativ in Java oder Scala basierend auf einer domänenspezifischen Sprache, oder mittels Spring basierter XML-Konfiguration definieren.\cite{CamelWiki} Es benützt URIs um direkt mit Transportprotokolle wie  HTTP, ActiveMQ, JMS, JBI, SCA, MINA und CXF zu arbeiten.

\section{Easy Bibliography}

\listoftables
\listoffigures
\printglossaries

\begin{thebibliography}{56}

\bibitem{EAIUniStuttgart}
   \textbf{Bettina Druckenmüller-Uni Stuttgart}\\
  \textit{http://elib.uni-stuttgart.de/opus/volltexte/2007/3049/pdf/DIP\_2583.pdf}
  \newline last used: 10.02.2015, 09:07

\bibitem{CamelWiki}
   \textbf{Wikipedia-Apache Camel}\\
  \textit{http://de.wikipedia.org/wiki/Apache\_Camel}
  \newline last used: 10.02.2015, 09:013
 \\\\
 \textbf{Quellen:}\\
 http://camel.apache.org/ - Apache Camel Homepage - 20.02.2015 \\
 
 
 
\end{thebibliography}
\end{document}
